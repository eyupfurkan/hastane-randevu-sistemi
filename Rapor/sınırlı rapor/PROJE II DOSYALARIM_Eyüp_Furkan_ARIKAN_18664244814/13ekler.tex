\section{EKLER}

\textbf{SQL VERİ TABANI KODLARI}

\textbf{Dosya 1:} Sql Veri Tabani:

\begin{lstlisting}
create database Hastane_Yonetim_ve_Randevu_Sistemi
use Hastane_Yonetim_ve_Randevu_Sistemi

create table Tbl_Yoneticiler(
Yoneticiid tinyint primary key identity(1,1) not null,
YoneticiTC varchar(11),
YoneticiAd varchar(30),
YoneticiSoyad varchar(20),
YoneticiSifre varchar(10)
)

create table Tbl_Branslar(
Bransid tinyint primary key identity(1,1) not null,
Yoneticiid tinyint foreign key references 
Tbl_Yoneticiler(Yoneticiid),
BransAd varchar(50)
)

create table Tbl_Doktorlar(
Doktorid tinyint primary key identity(1,1) not null,
Bransid tinyint foreign key references 
Tbl_Branslar(Bransid),
Yoneticiid tinyint foreign key references 
Tbl_Yoneticiler(Yoneticiid),
DoktorAd varchar(15),
DoktorSoyad varchar(15),
DoktorTC varchar(11),
DoktorSifre varchar(10),
)

create table Tbl_Sekreterler(
Sekreterid tinyint primary key identity(1,1) not null,
SekreterAd varchar(30),
SekreterSoyad varchar(20),
SekreterTC varchar(11),
SekreterSifre varchar(10),
)

create table Tbl_Duyurular(
Duyuruid smallint primary key identity(1,1) not null,
Yoneticiid tinyint foreign key references 
Tbl_Yoneticiler(Yoneticiid),
Doktorid tinyint foreign key references 
Tbl_Doktorlar(Doktorid),
Sekreterid tinyint foreign key references 
Tbl_Sekreterler(Sekreterid),
Duyuru varchar(255)
)

create table Tbl_Hastalar(
Hastaid smallint primary key identity(1,1) not null,
HastaAd varchar(30),
HastaSoyad varchar(20),
HastaTC varchar(11),
HastaTelefon varchar(15),
HastaSifre varchar(10),
HastaCinsiyet varchar(5)
)

create table Tbl_Musteri_Hizmetleri(
MusteriHizmetleriid smallint primary key identity(1,1) 
not null,
MusteriHizmetleriAd varchar(30),
MusteriHizmetleriSoyad varchar(20),
MusteriHizmetleriTC varchar(11),
MusteriHizmetleriSifre varchar(10)
)

create table Tbl_Randevular(
Randevuid int primary key identity(1,1) not null,
Bransid tinyint foreign key references 
Tbl_Branslar(Bransid),
Doktorid tinyint foreign key references 
Tbl_Doktorlar(Doktorid),
Hastaid smallint foreign key references 
Tbl_Hastalar(Hastaid),
MusteriHizmetleriid smallint foreign key references 
Tbl_Musteri_Hizmetleri(MusteriHizmetleriid),
RandevuTarih varchar(10),
RandevuSaat varchar(5),
RandevuDurum bit,
Sikayet varchar(255)
)
\end{lstlisting}

\textbf{C}\# \textbf{FORM PROGRAMI KODLARI}

\textbf{Dosya 2:} Sql Baglantisi Olusturma:

\begin{lstlisting}
using System;
using System.Collections.Generic;
using System.Linq;
using System.Text;
using System.Threading.Tasks;
using System.Data.SqlClient;

namespace Yonetim_Hastane
{
    class SqlBaglantisi
    {
        public SqlConnection baglanti()
        {
            SqlConnection baglan = new SqlConnection("Data
             Source=FURKAN\\SQLEXPRESS;Initial Catalog=
             Hastane_Yonetim_ve_Randevu_Sistemi;Integrated
              Security=True");
            baglan.Open();
            return baglan;
        }
    }
}
\end{lstlisting}

\textbf{Dosya 3:} Ilk Giris Formu:

\begin{lstlisting}
using System;
using System.Collections.Generic;
using System.ComponentModel;
using System.Data;
using System.Drawing;
using System.Linq;
using System.Text;
using System.Threading.Tasks;
using System.Windows.Forms;
using System.Data.SqlClient;

namespace Yonetim_Hastane
{
    public partial class FrmGirisler : Form
    {
        public FrmGirisler()
        {
            InitializeComponent();
        }

        SqlBaglantisi bgl = new SqlBaglantisi();

        private void button1_Click(object sender, 
        EventArgs e)
        {
            String secim = comboBox1.Text;

            if( secim.Equals("Hasta") )
            {
                SqlCommand komut = new SqlCommand("select *
                 from Tbl_Hastalar where HastaTC=@p1 and 
                 HastaSifre=@p2", bgl.baglanti());
                komut.Parameters.AddWithValue("@p1", 
                textBox1.Text);
                komut.Parameters.AddWithValue("@p2", 
                textBox2.Text);
                SqlDataReader dr = komut.ExecuteReader();
                if (dr.Read())
                {
                    FrmHastaDetay fr = new FrmHastaDetay();
                    fr.tc = textBox1.Text;
                    fr.Show();
                    this.Hide();
                }
                else
                {
                    MessageBox.Show("TC Kimlik No ya da 
                    şifre hatalıdır.", "Uyarı", 
                    MessageBoxButtons.OK, 
                    MessageBoxIcon.Warning);
                }
                bgl.baglanti().Close();
            } else if( secim.Equals("Doktor") )
            {
                SqlCommand komut = new SqlCommand("select *
                 from Tbl_Doktorlar where DoktorTC=@d1 and
                  DoktorSifre=@d2", bgl.baglanti());
                komut.Parameters.AddWithValue("@d1", 
                textBox1.Text);
                komut.Parameters.AddWithValue("@d2", 
                textBox2.Text);
                SqlDataReader dr = komut.ExecuteReader();
                if (dr.Read())
                {
                    FrmDoktorDetay fr = new 
                    FrmDoktorDetay();
                    fr.DoktorTCsi = textBox1.Text;
                    fr.Show();
                    this.Hide();
                }
                else
                {
                    MessageBox.Show("TC Kimlik No ya da 
                    şifre hatalıdır.", "Uyarı", 
                    MessageBoxButtons.OK, 
                    MessageBoxIcon.Warning);
                }
                bgl.baglanti().Close();
            } else if( secim.Equals("Sekreter") )
            {
                SqlCommand komut = new SqlCommand("select *
                 from Tbl_Sekreterler where SekreterTC=@p1 
                 and SekreterSifre=@p2", bgl.baglanti());
                komut.Parameters.AddWithValue("@p1", 
                textBox1.Text);
                komut.Parameters.AddWithValue("@p2", 
                textBox2.Text);
                SqlDataReader dr = komut.ExecuteReader();
                if (dr.Read())
                {
                    FrmSekreterDetay frs = new 
                    FrmSekreterDetay();
                    frs.SekreterTC = textBox1.Text;
                    frs.Show();
                    this.Hide();
                }
                else
                {
                    MessageBox.Show("TC Kimlik No ya da 
                    şifre hatalıdır.", "Uyarı", 
                    MessageBoxButtons.OK, 
                    MessageBoxIcon.Warning);
                }
                bgl.baglanti().Close();
            } else if( secim.Equals("Yönetici") )
            {
                SqlCommand komut = new SqlCommand("select *
                 from Tbl_Yoneticiler where YoneticiTC=@p1 
                 and YoneticiSifre=@p2", bgl.baglanti());
                komut.Parameters.AddWithValue("@p1", 
                textBox1.Text);
                komut.Parameters.AddWithValue("@p2", 
                textBox2.Text);
                SqlDataReader dr = komut.ExecuteReader();
                if (dr.Read())
                {
                    FrmYoneticiDetay frs = new 
                    FrmYoneticiDetay();
                    frs.YoneticiTC = textBox1.Text;
                    frs.Show();
                    this.Hide();
                }
                else
                {
                    MessageBox.Show("TC Kimlik No ya da 
                    şifre hatalıdır.", "Uyarı", 
                    MessageBoxButtons.OK, 
                    MessageBoxIcon.Warning);
                }
                bgl.baglanti().Close();
            } else if( secim.Equals("Müşteri Hizmetleri") )
            {
                SqlCommand komut = new SqlCommand("select *
                 from Tbl_Musteri_Hizmetleri where
                  MusteriHizmetleriTC=@p1 and
                   MusteriHizmetleriSifre=@p2", 
                   bgl.baglanti());
                komut.Parameters.AddWithValue("@p1", 
                textBox1.Text);
                komut.Parameters.AddWithValue("@p2", 
                textBox2.Text);
                SqlDataReader dr = komut.ExecuteReader();
                if (dr.Read())
                {
                    FrmMusteriHizmetleri frs = new
                     FrmMusteriHizmetleri();
                    frs.MusteriHizmetleriTC = textBox1.Text;
                    frs.Show();
                    this.Hide();
                }
                else
                {
                    MessageBox.Show("TC Kimlik No ya da 
                    şifre hatalıdır.", "Uyarı", 
                     MessageBoxButtons.OK, 
                     MessageBoxIcon.Warning);
                }
                bgl.baglanti().Close();
            }
        }

        private void button2_Click(object sender, 
        EventArgs e)
        {
            FrmHastaKayit fr = new FrmHastaKayit();
            fr.Show();
        }
    }
}

\end{lstlisting}

\textbf{Dosya 4:} Hasta Kayıt Formu:

\begin{lstlisting}
using System;
using System.Collections.Generic;
using System.ComponentModel;
using System.Data;
using System.Drawing;
using System.Linq;
using System.Text;
using System.Threading.Tasks;
using System.Windows.Forms;
using System.Data.SqlClient;

namespace Yonetim_Hastane
{
    public partial class FrmHastaKayit : Form
    {
        public FrmHastaKayit()
        {
            InitializeComponent();
        }

        SqlBaglantisi bgl = new SqlBaglantisi();

        private void BtnKayitYap_Click(object sender, 
        EventArgs e)
        {
            if ( TCKontrol(MskTC.Text) ) {
                SqlCommand komut = new SqlCommand("insert 
                into
                 Tbl_Hastalar(HastaAd,HastaSoyad,HastaTC,
                 HastaTelefon,HastaSifre,HastaCinsiyet)
                  values(@p1,@p2,@p3,@p4,@p5,@p6)", 
                  bgl.baglanti());
                komut.Parameters.AddWithValue("@p1", 
                TxtAd.Text);
                komut.Parameters.AddWithValue("@p2", 
                TxtSoyad.Text);
                komut.Parameters.AddWithValue("@p3", 
                MskTC.Text);
                komut.Parameters.AddWithValue("@p4", 
                MskTelefon.Text);
                komut.Parameters.AddWithValue("@p5", 
                TxtSifre.Text);
                komut.Parameters.AddWithValue("@p6", 
                CmbCinsiyet.Text);
                komut.ExecuteNonQuery();
                bgl.baglanti().Close();
                MessageBox.Show("Kaydınız gerçekleşmiştir. 
                Şifreniz: " + TxtSifre.Text, "Bilgi",
                 MessageBoxButtons.OK, 
                 MessageBoxIcon.Information);
            }
        }

        public static Boolean TCKontrol(String TC)
        {
            string tcNo = TC;
            int toplam = 0; int toplam2 = 0; 
            int toplam3 = 0;
            if (tcNo.Length == 11)
            {
                if (Convert.ToInt32(tcNo[0].ToString()) != 
                0) //tc kimlik numaranın ilk hanesi 0 
                değilse
                {
                    for (int i = 0; i < 10; i++)
                    {
                        toplam = toplam +
                         Convert.ToInt32(tcNo[i].
                         ToString());
                        if (i % 2 == 0)
                        {
                            if (i != 10)
                            {
                                toplam2 = toplam2 +
                                 Convert.ToInt32(tcNo[i].
                                 ToString()); // 7 ile 
                                 çarpılacak sayıları 
                                 topluyoruz
                            }

                        }
                        else
                        {
                            if (i != 9)
                            {
                                toplam3 = toplam3 +
                                 Convert.ToInt32(tcNo[i].
                                 ToString());
                            }
                        }
                    }
                }
                else
                {
                    MessageBox.Show("Tc Kimlik Numaranızın 
                    ilk hanesi 0 olamaz.");
                    return false;
                }
            }
            else
            {
                MessageBox.Show("Tc Kimlik Numarınız 11 
                haneli olmak zorunda.Eksik ya da fazla 
                değer girdiniz.");
                return false;

            }
            if (((toplam2 * 7) - toplam3) % 10 ==
             Convert.ToInt32(tcNo[9].ToString()) && 
             toplam % 10 == Convert.ToInt32(tcNo[10].
             ToString()))
            {
                return true;
            }
            else
            {
                MessageBox.Show("Tc Kimlik Numarası 
                Yanlış!");
                return false;
            }
        }

    }
}
\end{lstlisting}

\textbf{Dosya 5:} Hasta Detay Formu:

\begin{lstlisting}
using System;
using System.Collections.Generic;
using System.ComponentModel;
using System.Data;
using System.Drawing;
using System.Linq;
using System.Text;
using System.Threading.Tasks;
using System.Windows.Forms;
using System.Data.SqlClient;

namespace Yonetim_Hastane
{
    public partial class FrmHastaDetay : Form
    {
        public FrmHastaDetay()
        {
            InitializeComponent();
        }

        public string tc;

        SqlBaglantisi bgl = new SqlBaglantisi();

        private void FrmHastaDetay_Load(object sender, 
        EventArgs e)
        {
            LblTC.Text = tc;
            //Ad Soyad Çekme
            SqlCommand komut = new SqlCommand("select 
            HastaAd,HastaSoyad from Tbl_Hastalar where 
            HastaTC=@p1", bgl.baglanti());
            komut.Parameters.AddWithValue("@p1", tc);
            SqlDataReader dr = komut.ExecuteReader();
            while (dr.Read())
            {
                LblAdSoyad.Text = dr[0] + " " + dr[1];
            }
            bgl.baglanti().Close();

            /*//Randevu Geçmişi
            DataTable dt = new DataTable();
            SqlDataAdapter da = new SqlDataAdapter("select 
            * from Tbl_Randevular where Hastaid="+tc,
            bgl.baglanti());
            da.Fill(dt);
            dataGridView1.DataSource = dt; */

            //Branş Çekme

            SqlCommand komut2 = new SqlCommand("select 
            BransAd from Tbl_Branslar",bgl.baglanti());
            SqlDataReader dr2 = komut2.ExecuteReader();
            while (dr2.Read())
            {
                CmbBrans.Items.Add(dr2[0]);
            }
            bgl.baglanti().Close();

            DataTable dt2 = new DataTable();
            SqlDataAdapter da2 = new SqlDataAdapter("SELECT
             Tbl_Randevular.Randevuid, Tbl_Doktorlar.
             DoktorAd, Tbl_Doktorlar.DoktorSoyad,
              Tbl_Randevular.RandevuTarih, Tbl_Randevular.
              RandevuSaat, Tbl_Randevular.RandevuDurum,
               Tbl_Randevular.Sikayet FROM Tbl_Randevular,
                Tbl_Doktorlar WHERE Hastaid IN (select 
                Hastaid from Tbl_Hastalar where HastaTC = 
                '" + tc + "' and RandevuDurum='True') AND
                 Tbl_Randevular.Doktorid = 
                 Tbl_Doktorlar.Doktorid", bgl.baglanti());
            da2.Fill(dt2);
            dataGridView1.DataSource = dt2;
        }

        private void CmbBrans_SelectedIndexChanged(object 
        sender, EventArgs e)
        {
            CmbDoktor.Items.Clear();
            SqlCommand komut3 = new SqlCommand("SELECT
             DoktorAd,DoktorSoyad FROM Tbl_Doktorlar 
             WHERE Bransid IN (select Bransid from 
             tbl_Branslar where BransAd LIKE @p1)", 
             bgl.baglanti());
            komut3.Parameters.AddWithValue("@p1", 
            CmbBrans.Text);
            SqlDataReader dr3 = komut3.ExecuteReader();
            while (dr3.Read())
            {
                CmbDoktor.Items.Add(dr3[0] + " " + dr3[1]);
            }
            bgl.baglanti().Close();
        }

        private void CmbDoktor_SelectedIndexChanged(object 
        sender, EventArgs e)
        {
            DataTable dt = new DataTable();
            SqlDataAdapter da = new SqlDataAdapter("select
             Randevuid,RandevuTarih,RandevuSaat from 
             Tbl_Randevular where Bransid IN (select 
             Bransid FROM Tbl_Branslar WHERE BransAd LIKE 
             '" + CmbBrans.Text + "') and Doktorid IN 
             (select Doktorid FROM Tbl_Doktorlar WHERE 
             DoktorAd + ' ' + DoktorSoyad LIKE '" + 
             CmbDoktor.Text + "' and RandevuDurum='False') 
             ", bgl.baglanti());
            da.Fill(dt);
            dataGridView2.DataSource = dt;
        }

        private void LnkBilgiDuzenle_LinkClicked(object 
        sender, LinkLabelLinkClickedEventArgs e)
        {
            FrmBilgiDuzenle fr = new FrmBilgiDuzenle();
            fr.TCno = LblTC.Text;
            fr.Show();
        }

        private void dataGridView2_CellClick(object sender,
         DataGridViewCellEventArgs e)
        {
            int secilen = dataGridView2.SelectedCells[0].
            RowIndex;
            Txtid.Text = dataGridView2.Rows[secilen].
            Cells[0].Value.ToString();
        }

        public string RandevuTarihi, RandevuSaati;

        private void dataGridView2_CellContentClick(object 
        sender, DataGridViewCellEventArgs e)
        {

        }

        private void button1_Click(object sender, EventArgs 
        e)
        {
            SqlCommand komut = new SqlCommand("UPDATE 
            Tbl_Randevular SET RandevuDurum='False', 
            Hastaid=null WHERE Randevuid='" + Txtid.Text + 
            "' ", bgl.baglanti());
            komut.ExecuteNonQuery();
            bgl.baglanti().Close();
            MessageBox.Show("Randevu iptal edildi.", 
            "Bilgi", MessageBoxButtons.OK, 
            MessageBoxIcon.Information);

            TabloGuncelle();

        }

        private void dataGridView1_CellClick(object sender,
         DataGridViewCellEventArgs e)
        {
            int secilen = dataGridView1.SelectedCells[0].
            RowIndex;
            Txtid.Text = dataGridView1.Rows[secilen].
            Cells[0].Value.ToString();
        }

        private void TabloGuncelle()
        {
            //Tablo Güncelleme
            DataTable dt2 = new DataTable();
            SqlDataAdapter da2 = new SqlDataAdapter("SELECT
             Tbl_Randevular.Randevuid, Tbl_Doktorlar.
             DoktorAd, Tbl_Doktorlar.DoktorSoyad,
              Tbl_Randevular.RandevuTarih, Tbl_Randevular.
              RandevuSaat, Tbl_Randevular.RandevuDurum,
               Tbl_Randevular.Sikayet FROM Tbl_Randevular,
                Tbl_Doktorlar WHERE Hastaid IN (select 
                Hastaid from Tbl_Hastalar where HastaTC = 
                '" + tc + "' and RandevuDurum='True') AND
                 Tbl_Randevular.Doktorid = Tbl_Doktorlar.
                 Doktorid", bgl.baglanti());
            da2.Fill(dt2);
            dataGridView1.DataSource = dt2;
            DataTable dt = new DataTable();
            SqlDataAdapter da = new SqlDataAdapter("select
             Randevuid,RandevuTarih,RandevuSaat from 
             Tbl_Randevular where Bransid IN (select 
             Bransid FROM Tbl_Branslar WHERE BransAd LIKE 
             '" + CmbBrans.Text + "') and Doktorid IN 
             (select Doktorid FROM Tbl_Doktorlar WHERE 
             DoktorAd + ' ' + DoktorSoyad LIKE '" + 
             CmbDoktor.Text + "' and RandevuDurum='False') 
             ", bgl.baglanti());
            da.Fill(dt);
            dataGridView2.DataSource = dt;
        }

        private void button2_Click(object sender, 
        EventArgs e)
        {
            Application.Exit();
        }

        private void BtnRandevuAl_Click(object sender, 
        EventArgs e)
        {
            SqlCommand komut = new SqlCommand("UPDATE 
            Tbl_Randevular SET RandevuDurum='True',
            Hastaid=(SELECT Hastaid FROM Tbl_Hastalar 
            WHERE HastaTC ='" + tc + "'),Sikayet=@r2 
            WHERE Randevuid='" + Txtid.Text + "' ", 
            bgl.baglanti());
            komut.Parameters.AddWithValue("@r2", 
            RchSikayet.Text);
            komut.ExecuteNonQuery();
            bgl.baglanti().Close();
            MessageBox.Show("Randevu oluşturuldu.", 
            "Bilgi", MessageBoxButtons.OK, 
            MessageBoxIcon.Information);

            TabloGuncelle();
            


            /*SqlCommand komut = new SqlCommand("insert 
            into Tbl_Randevular
             (RandevuDurum,Hastaid,Sikayet,Doktorid,
             Bransid) SELECT 1,(SELECT Hastaid FROM 
             Tbl_Hastalar WHERE HastaTC ='"+ tc +"'), 
             @r2, (SELECT Doktorid from Tbl_Doktorlar 
             WHERE DoktorAd+' '+DoktorSoyad ='"+ 
             CmbDoktor.Text + "'),(SELECT Bransid from 
             Tbl_Branslar WHERE BransAd ='" + 
             CmbBrans.Text + "')", bgl.baglanti());
            komut.Parameters.AddWithValue("@r2", 
            RchSikayet.Text);
            komut.Parameters.AddWithValue("@r3", 
            Txtid.Text);
            komut.ExecuteNonQuery();
            bgl.baglanti().Close();

            //dataGridView2_CellClick özelliği çalışmadığı 
            için Randevuid gelmiyor ve aşağıdaki kodda 
            veri gelirken hata veriyor.
            SqlCommand komut2 = new SqlCommand("select
             RandevuTarih,RandevuSaat from Tbl_Randevular 
             where Randevuid=@r4", bgl.baglanti());
            komut2.Parameters.AddWithValue("@r4", 
            Txtid.Text);
            SqlDataReader dr = komut2.ExecuteReader();
            RandevuTarihi = dr[1].ToString();
            RandevuSaati = dr[2].ToString();
            MessageBox.Show(CmbBrans.Text + " bölümünden " 
            + CmbDoktor.Text + " doktordan \n" + 
            RandevuTarihi + " tarihinde " + RandevuSaati +
             " ve saatinde randevunuz alınmıştır.", 
             "Bilgi", MessageBoxButtons.OK, 
             MessageBoxIcon.Information);
            bgl.baglanti().Close();*/
            //Buraya kadar!
        }
    }
}
\end{lstlisting}

\textbf{Dosya 6:} Bilgi Düzenle Formu:

\begin{lstlisting}
using System;
using System.Collections.Generic;
using System.ComponentModel;
using System.Data;
using System.Drawing;
using System.Linq;
using System.Text;
using System.Threading.Tasks;
using System.Windows.Forms;
using System.Data.SqlClient;

namespace Yonetim_Hastane
{
    public partial class FrmBilgiDuzenle : Form
    {
        public FrmBilgiDuzenle()
        {
            InitializeComponent();
        }

        public string TCno;

        SqlBaglantisi bgl = new SqlBaglantisi();
        private void FrmBilgiDuzenle_Load(object sender, 
        EventArgs e)
        {
            MskTC.Text = TCno;
            SqlCommand komut = new SqlCommand("select * 
            from Tbl_Hastalar where HastaTC=@p1", 
            bgl.baglanti());
            komut.Parameters.AddWithValue("@p1", 
            MskTC.Text);
            SqlDataReader dr = komut.ExecuteReader();
            while (dr.Read())
            {
                TxtAd.Text = dr[1].ToString();
                TxtSoyad.Text = dr[2].ToString();
                MskTelefon.Text = dr[4].ToString();
                TxtSifre.Text = dr[5].ToString();
                CmbCinsiyet.Text = dr[6].ToString();
            }
            bgl.baglanti().Close();
        }

        private void BtnBilgiGuncelle_Click(object sender, 
        EventArgs e)
        {
            if (FrmHastaKayit.TCKontrol(MskTC.Text)) {
                SqlCommand komut2 = new SqlCommand("update 
                Tbl_Hastalar set HastaAd=@p1, 
                HastaSoyad=@p2, HastaTelefon=@p3, 
                HastaSifre=@p4, HastaCinsiyet=@p5 where 
                HastaTC=@p6", bgl.baglanti());
                komut2.Parameters.AddWithValue("@p1", 
                TxtAd.Text);
                komut2.Parameters.AddWithValue("@p2", 
                TxtSoyad.Text);
                komut2.Parameters.AddWithValue("@p3", 
                MskTelefon.Text);
                komut2.Parameters.AddWithValue("@p4", 
                TxtSifre.Text);
                komut2.Parameters.AddWithValue("@p5", 
                CmbCinsiyet.Text);
                komut2.Parameters.AddWithValue("@p6", 
                MskTC.Text);
                komut2.ExecuteNonQuery();
                bgl.baglanti().Close();
                MessageBox.Show("Bilgileriniz güncellendi."
                , "Bilgi", MessageBoxButtons.OK,
                 MessageBoxIcon.Information);

            }
        }
    }
}
\end{lstlisting}

\textbf{Dosya 7:} Doktor Detay Formu:

\begin{lstlisting}
using System;
using System.Collections.Generic;
using System.ComponentModel;
using System.Data;
using System.Drawing;
using System.Linq;
using System.Text;
using System.Threading.Tasks;
using System.Windows.Forms;
using System.Data.SqlClient;

namespace Yonetim_Hastane
{
    public partial class FrmDoktorDetay : Form
    {
        public FrmDoktorDetay()
        {
            InitializeComponent();
        }

        SqlBaglantisi bgl = new SqlBaglantisi();

        public string DoktorTCsi;
        private void FrmDoktorDetay_Load(object sender, 
        EventArgs e)
        {
            LblTC.Text = DoktorTCsi;
            //Doktor Ad Soyad Çekme
            SqlCommand komut = new SqlCommand("select
             DoktorAd,DoktorSoyad,Doktorid from 
             Tbl_Doktorlar where DoktorTC=@d1", 
             bgl.baglanti());
            komut.Parameters.AddWithValue("@d1", 
            LblTC.Text);
            SqlDataReader dr = komut.ExecuteReader();

            //Doktor Ad, Soyad ve Randevu Listesi
            while (dr.Read())
            {
                String DoktorAd = dr[0].ToString();
                String DoktorSoyad = dr[1].ToString();
                LblAdSoyad.Text = DoktorAd + " " + 
                DoktorSoyad;
                DataTable dt = new DataTable();
                SqlDataAdapter da = new SqlDataAdapter
                ("select Tbl_Randevular.Randevuid, 
                Tbl_Hastalar.HastaAd, 
                Tbl_Hastalar.HastaSoyad, 
                Tbl_Randevular.RandevuTarih, 
                Tbl_Randevular.RandevuSaat, 
                Tbl_Randevular.RandevuDurum, 
                Tbl_Randevular.Sikayet from 
                Tbl_Randevular, Tbl_Hastalar where 
                Doktorid='" + dr[2].ToString() + "' 
                and RandevuDurum='True' AND 
                Tbl_Randevular.Hastaid = 
                Tbl_Hastalar.Hastaid", bgl.baglanti());
                da.Fill(dt);
                dataGridView1.DataSource = dt;
            }
            bgl.baglanti().Close();
        }

        private void BtnGuncelle_Click(object sender, 
        EventArgs e)
        {
            FrmDoktorBilgiDuzenle dbd = new 
            FrmDoktorBilgiDuzenle();
            dbd.DoktorTCsidir = LblTC.Text;
            dbd.Show();
        }

        private void BtnDuyurular_Click(object sender, 
        EventArgs e)
        {
            FrmDuyurular d = new FrmDuyurular();
            d.Show();
        }

        private void BtnCikis_Click(object sender, 
        EventArgs e)
        {
            Application.Exit();
        }

        private void dataGridView1_CellClick(object sender,
         DataGridViewCellEventArgs e)
        {
            int secilen = dataGridView1.SelectedCells[0].
            RowIndex;
            RchSikayet.Text = dataGridView1.Rows[secilen].
            Cells[4].Value.ToString();
            txtRandevuid.Text = dataGridView1.
            Rows[secilen].Cells[0].Value.ToString();
        }

        private void dataGridView1_CellContentClick(object 
        sender, DataGridViewCellEventArgs e)
        {

        }

        private void btnTamamla_Click(object sender, 
        EventArgs e)
        {
            SqlCommand komut2 = new SqlCommand("delete from
             Tbl_Randevular where Randevuid=@p1", 
             bgl.baglanti());
            komut2.Parameters.AddWithValue("@p1", 
            txtRandevuid.Text);
            komut2.ExecuteNonQuery();
            bgl.baglanti().Close();
            MessageBox.Show("Randevu Tamamlandı.", "Bilgi",
             MessageBoxButtons.OK, 
             MessageBoxIcon.Information);
            TabloGuncelle();
        }

        private void TabloGuncelle()
        {
            LblTC.Text = DoktorTCsi;
            //Doktor Ad Soyad Çekme
            SqlCommand komut = new SqlCommand("select
             DoktorAd,DoktorSoyad,Doktorid from 
             Tbl_Doktorlar where DoktorTC=@d1", 
             bgl.baglanti());
            komut.Parameters.AddWithValue("@d1", 
            LblTC.Text);
            SqlDataReader dr = komut.ExecuteReader();

            while (dr.Read())
            {
                String DoktorAd = dr[0].ToString();
                String DoktorSoyad = dr[1].ToString();
                LblAdSoyad.Text = DoktorAd + " " + 
                DoktorSoyad;
                DataTable dt = new DataTable();
                SqlDataAdapter da = new SqlDataAdapter
                ("select Tbl_Randevular.Randevuid, 
                Tbl_Hastalar.HastaAd, 
                Tbl_Hastalar.HastaSoyad, 
                Tbl_Randevular.RandevuTarih, 
                Tbl_Randevular.RandevuSaat, 
                Tbl_Randevular.RandevuDurum, 
                Tbl_Randevular.Sikayet from 
                Tbl_Randevular, Tbl_Hastalar where 
                Doktorid='" + dr[2].ToString() + "' and
                 RandevuDurum='True' AND Tbl_Randevular.
                 Hastaid = Tbl_Hastalar.Hastaid", 
                 bgl.baglanti());
                da.Fill(dt);
                dataGridView1.DataSource = dt;
            }
            bgl.baglanti().Close();
        }
    }
}
\end{lstlisting}

\textbf{Dosya 8:} Duyurular Formu:

\begin{lstlisting}
using System;
using System.Collections.Generic;
using System.ComponentModel;
using System.Data;
using System.Drawing;
using System.Linq;
using System.Text;
using System.Threading.Tasks;
using System.Windows.Forms;
using System.Data.SqlClient;

namespace Yonetim_Hastane
{
    public partial class FrmDuyurular : Form
    {
        public FrmDuyurular()
        {
            InitializeComponent();
        }

        SqlBaglantisi bgl = new SqlBaglantisi();

        private void FrmDuyurular_Load(object sender, 
        EventArgs e)
        {
            DataTable dt = new DataTable();
            SqlDataAdapter da = new SqlDataAdapter("select
             Duyuruid,Duyuru from Tbl_Duyurular",
             bgl.baglanti());
            da.Fill(dt);
            dataGridView1.DataSource = dt;
        }
    }
}
\end{lstlisting}

\textbf{Dosya 9:} Sekreter Detay Formu:

\begin{lstlisting}
using System;
using System.Collections.Generic;
using System.ComponentModel;
using System.Data;
using System.Drawing;
using System.Linq;
using System.Text;
using System.Threading.Tasks;
using System.Windows.Forms;
using System.Data.SqlClient;

namespace Yonetim_Hastane
{
    public partial class FrmSekreterDetay : Form
    {
        public FrmSekreterDetay()
        {
            InitializeComponent();
        }

        public string SekreterTC;

        SqlBaglantisi bgl = new SqlBaglantisi();

        private void FrmSekreterDetay_Load(object sender, 
        EventArgs e)
        {
            LblTC.Text = SekreterTC;
            //Sekreter AdSoyad Çekme
            SqlCommand komut = new SqlCommand("select
             SekreterAd,SekreterSoyad from Tbl_Sekreterler 
             where SekreterTC=@p1",bgl.baglanti());
            komut.Parameters.AddWithValue("@p1", 
            LblTC.Text);
            SqlDataReader dr1 = komut.ExecuteReader();
            while (dr1.Read())
            {
                LblAdSoyad.Text = dr1[0].ToString() + " " +
                 dr1[1].ToString();
            }
            bgl.baglanti().Close();

            //Randevuları Datagridview1'e aktarma
            DataTable dt = new DataTable();
            SqlDataAdapter da = new SqlDataAdapter("select
             Tbl_Randevular.Randevuid, 
             Tbl_Doktorlar.DoktorAd, 
             Tbl_Doktorlar.DoktorSoyad, 
             Tbl_Branslar.BransAd, 
             Tbl_Randevular.RandevuTarih, 
             Tbl_Randevular.RandevuSaat, 
             Tbl_Randevular.RandevuDurum, 
             Tbl_Randevular.Sikayet from Tbl_Randevular, 
             Tbl_Doktorlar, Tbl_Branslar WHERE 
             Tbl_Randevular.Doktorid = Tbl_Doktorlar.
             Doktorid AND Tbl_Randevular.Bransid = 
             Tbl_Branslar.Bransid", bgl.baglanti());
            da.Fill(dt);
            dataGridView1.DataSource = dt;

            //Branşı Combobox'a aktarma
            SqlCommand komut2 = new SqlCommand("select 
            BransAd from Tbl_Branslar",bgl.baglanti());
            SqlDataReader dr2 = komut2.ExecuteReader();
            while (dr2.Read())
            {
                CmbBrans.Items.Add(dr2[0]);
            }
            bgl.baglanti().Close();
        }

        private void TabloyuGuncelle()
        {
            DataTable dt = new DataTable();
            SqlDataAdapter da = new SqlDataAdapter("select
             Tbl_Randevular.Randevuid, 
             Tbl_Doktorlar.DoktorAd, 
             Tbl_Doktorlar.DoktorSoyad, 
             Tbl_Branslar.BransAd, 
             Tbl_Randevular.RandevuTarih, 
             Tbl_Randevular.RandevuSaat, 
             Tbl_Randevular.RandevuDurum, 
             Tbl_Randevular.Sikayet from Tbl_Randevular, 
             Tbl_Doktorlar, Tbl_Branslar WHERE 
             Tbl_Randevular.Doktorid = Tbl_Doktorlar
             .Doktorid AND Tbl_Randevular.Bransid =
              Tbl_Branslar.Bransid", bgl.baglanti());
            da.Fill(dt);
            dataGridView1.DataSource = dt;
        }

        private void BtnKaydet_Click(object sender, 
        EventArgs e)
        {
            SqlCommand komutkaydet = new SqlCommand("insert
             into
              Tbl_Randevular(RandevuTarih,RandevuSaat,
              Bransid,Doktorid,RandevuDurum) values(@p1,@p2,
              (SELECT Bransid FROM Tbl_Branslar WHERE 
              BransAd='"+ CmbBrans.Text + "'),(SELECT 
              Doktorid FROM Tbl_Doktorlar WHERE DoktorAd+'
               '+DoktorSoyad='" + CmbDoktor.Text + "'),
               'False')", bgl.baglanti());
            komutkaydet.Parameters.AddWithValue("@p1", 
            MskTarih.Text);
            komutkaydet.Parameters.AddWithValue("@p2", 
            MskSaat.Text);
            komutkaydet.Parameters.AddWithValue("@p3",
            CmbBrans.Text);
            komutkaydet.Parameters.AddWithValue("@p4",
            CmbDoktor.Text);
            komutkaydet.ExecuteNonQuery();
            bgl.baglanti().Close();
            MessageBox.Show("Randevu oluşturuldu.","Bilgi",
            MessageBoxButtons.OK,
            MessageBoxIcon.Information);
            TabloyuGuncelle();
        }

        private void CmbBrans_SelectedIndexChanged(object 
        sender, EventArgs e)
        {
            CmbDoktor.Items.Clear();
            SqlCommand komut = new SqlCommand("select
             DoktorAd,DoktorSoyad from Tbl_Doktorlar where 
             Bransid = (SELECT Bransid FROM Tbl_Branslar 
             WHERE BransAd='"+ CmbBrans.Text + "')", 
             bgl.baglanti());
            SqlDataReader dr = komut.ExecuteReader();
            while (dr.Read())
            {
                CmbDoktor.Items.Add(dr[0] + " " + dr[1]);
            }
            bgl.baglanti().Close();
        }

        private void BtnDuyuruOlustur_Click(object sender, 
        EventArgs e)
        {
            SqlCommand komut = new SqlCommand("insert into
             Tbl_Duyurular(Duyuru) values(@d1)", 
             bgl.baglanti());
            komut.Parameters.AddWithValue("@d1", 
            RchDuyuru.Text);
            komut.ExecuteNonQuery();
            bgl.baglanti().Close();
            MessageBox.Show("Duyuru oluşturuldu.","Bilgi",
            MessageBoxButtons.OK,
            MessageBoxIcon.Information);
        }

        private void BtnDoktorPaneli_Click(object sender, 
        EventArgs e)
        {
            FrmDoktorPaneli drp = new FrmDoktorPaneli();
            drp.Show();
        }

        private void BtnBransPaneli_Click(object sender, 
        EventArgs e)
        {
            FrmBransPaneli frb = new FrmBransPaneli();
            frb.Show();
        }

        private void BtnRandevuListe_Click(object sender, 
        EventArgs e)
        {
            FrmRandevuListesi rl = new FrmRandevuListesi();
            rl.Show();
        }

        private void button1_Click(object sender, 
        EventArgs e)
        {
            FrmDuyurular frd = new FrmDuyurular();
            frd.Show();
        }

        private void BtnGuncelle_Click(object sender, 
        EventArgs e)
        {
            //SqlCommand komut2 = new SqlCommand("update 
            Tbl_Randevular set RandevuTarih=@p1, 
            RandevuSaat=@p2, Bransid=(SELECT Bransid FROM 
            Tbl_Branslar WHERE BransAd=@p3), Doktorid=
            (SELECT Doktorid FROM Tbl_Doktorlar WHERE 
            DoktorAd=@p4 and DoktorSoyad= @p5) where 
            Randevuid=@p6", bgl.baglanti());

            SqlCommand komut2 = new SqlCommand("update 
            Tbl_Randevular set RandevuTarih=@p1, 
            RandevuSaat=@p2 where Randevuid=@p6", 
            bgl.baglanti());
            komut2.Parameters.AddWithValue("@p1", 
            MskTarih.Text);
            komut2.Parameters.AddWithValue("@p2", 
            MskSaat.Text);
            komut2.Parameters.AddWithValue("@p3", 
            CmbBrans.Text);
            komut2.Parameters.AddWithValue("@p4", 
            DoktorAd);
            komut2.Parameters.AddWithValue("@p5", 
            DoktorSoyad);
            komut2.Parameters.AddWithValue("@p6", 
            Txtid.Text);
            komut2.ExecuteNonQuery();
            bgl.baglanti().Close();
            MessageBox.Show("Bilgileriniz güncellendi.", 
            "Bilgi", MessageBoxButtons.OK, 
            MessageBoxIcon.Information);
            TabloyuGuncelle();
        }

        String DoktorAd, DoktorSoyad;
        private void dataGridView1_CellClick(object sender,
         DataGridViewCellEventArgs e)
        {
            int secilen = dataGridView1.SelectedCells[0].
            RowIndex;
            Txtid.Text = dataGridView1.Rows[secilen].
            Cells[0].Value.ToString();
            MskTarih.Text = dataGridView1.Rows[secilen].
            Cells[4].Value.ToString();
            MskSaat.Text = dataGridView1.Rows[secilen].
            Cells[5].Value.ToString();
            CmbBrans.Text = dataGridView1.Rows[secilen].
            Cells[3].Value.ToString();
            CmbDoktor.Text = dataGridView1.Rows[secilen].
            Cells[1].Value.ToString() + " " +
             dataGridView1.Rows[secilen].
            Cells[2].Value.ToString();
            DoktorAd = dataGridView1.Rows[secilen].
            Cells[1].Value.ToString();
            DoktorSoyad = dataGridView1.Rows[secilen].
            Cells[2].Value.ToString();
        }

        private void button2_Click(object sender, 
        EventArgs e)
        {
            Application.Exit();
        }

        private void button3_Click(object sender, 
        EventArgs e)
        {
            FrmDoktorVeBranslar frd = new 
            FrmDoktorVeBranslar();
            frd.Show();
        }

        private void button4_Click(object sender, 
        EventArgs e)
        {
            SqlCommand komut = new SqlCommand("DELETE FROM 
            Tbl_Randevular WHERE Randevuid='" + Txtid.Text 
            + "'", bgl.baglanti());
            komut.ExecuteNonQuery();
            bgl.baglanti().Close();
            MessageBox.Show("Randevu silindi.", "Bilgi",
             MessageBoxButtons.OK, 
             MessageBoxIcon.Information);
            TabloyuGuncelle();
        }
    }
}
\end{lstlisting}

\textbf{Dosya 10:} Randevu Listesi Formu:

\begin{lstlisting}
using System;
using System.Collections.Generic;
using System.ComponentModel;
using System.Data;
using System.Drawing;
using System.Linq;
using System.Text;
using System.Threading.Tasks;
using System.Windows.Forms;
using System.Data.SqlClient;

namespace Yonetim_Hastane
{
    public partial class FrmRandevuListesi : Form
    {
        public FrmRandevuListesi()
        {
            InitializeComponent();
        }

        SqlBaglantisi bgl = new SqlBaglantisi();

        private void FrmRandevuListesi_Load(object sender, 
        EventArgs e)
        {
            DataTable dt = new DataTable();
            SqlDataAdapter da = new SqlDataAdapter("select
             Tbl_Randevular.Randevuid, Tbl_Hastalar.
             HastaAd, Tbl_Hastalar.HastaSoyad, 
             Tbl_Branslar.BransAd, Tbl_Doktorlar.
             DoktorAd, Tbl_Doktorlar.DoktorSoyad,
              Tbl_Randevular.RandevuTarih, Tbl_Randevular.
              RandevuSaat, Tbl_Randevular.RandevuDurum,
               Tbl_Randevular.Sikayet   FROM 
               Tbl_Randevular, Tbl_Branslar, Tbl_Doktorlar,
                Tbl_Hastalar WHERE Tbl_Randevular.Bransid =
                 Tbl_Branslar.Bransid AND Tbl_Randevular.
                 Doktorid = Tbl_Doktorlar.Doktorid AND
                  Tbl_Randevular.Hastaid = 
                  Tbl_Hastalar.Hastaid ", bgl.baglanti());
            da.Fill(dt);
            dataGridView1.DataSource = dt;
        }

        public int secilen;
        private void dataGridView1_CellDoubleClick(object 
        sender, DataGridViewCellEventArgs e)
        {
            secilen = dataGridView1.SelectedCells[0].
            RowIndex;
        }
    }
}
\end{lstlisting}

\textbf{Dosya 11:} Doktor ve Branşlar Formu:

\begin{lstlisting}
using System;
using System.Collections.Generic;
using System.ComponentModel;
using System.Data;
using System.Drawing;
using System.Linq;
using System.Text;
using System.Threading.Tasks;
using System.Windows.Forms;
using System.Data.SqlClient;

namespace Yonetim_Hastane
{
    public partial class FrmDoktorVeBranslar : Form
    {
        public FrmDoktorVeBranslar()
        {
            InitializeComponent();
        }

        SqlBaglantisi bgl = new SqlBaglantisi();

        private void DoktorVeBranslar_Load(object sender, 
        EventArgs e)
        {
            DataTable dt1 = new DataTable();
            SqlDataAdapter da1 = new SqlDataAdapter("select
             Doktorid,DoktorAd,DoktorSoyad,BransAd from 
             Tbl_Doktorlar, Tbl_Branslar where
              Tbl_Branslar.Bransid=Tbl_Doktorlar.Bransid",
               bgl.baglanti());
            da1.Fill(dt1);
            dataGridView1.DataSource = dt1;

            DataTable dt2 = new DataTable();
            SqlDataAdapter da2 = new SqlDataAdapter("select
             Bransid,BransAd from Tbl_Branslar", 
             bgl.baglanti());
            da2.Fill(dt2);
            dataGridView2.DataSource = dt2;
        }
    }
}
\end{lstlisting}

\textbf{Dosya 12:} Müşteri Hizmetleri Formu:

\begin{lstlisting}
using System;
using System.Collections.Generic;
using System.ComponentModel;
using System.Data;
using System.Drawing;
using System.Linq;
using System.Text;
using System.Threading.Tasks;
using System.Windows.Forms;
using System.Data.SqlClient;

namespace Yonetim_Hastane
{
    public partial class FrmMusteriHizmetleri : Form
    {
        public FrmMusteriHizmetleri()
        {
            InitializeComponent();
        }

        SqlBaglantisi bgl = new SqlBaglantisi();

        public string MusteriHizmetleriTC;
        private void FrmMusteriHizmetleri_Load(object 
        sender, EventArgs e)
        {
            SqlCommand komut2 = new SqlCommand("select 
            BransAd from Tbl_Branslar", bgl.baglanti());
            SqlDataReader dr2 = komut2.ExecuteReader();
            while (dr2.Read())
            {
                CmbBrans.Items.Add(dr2[0]);
            }
            bgl.baglanti().Close();

            lblMusteriHizmetleriTC.Text = 
            MusteriHizmetleriTC;

            //Musteri Hizmetleri AdSoyad Çekme
            SqlCommand komut = new SqlCommand("select
             MusteriHizmetleriAd, MusteriHizmetleriSoyad 
             from Tbl_Musteri_Hizmetleri where 
            MusteriHizmetleriTC=@p1", bgl.baglanti());
            komut.Parameters.AddWithValue("@p1", 
            MusteriHizmetleriTC);
            SqlDataReader dr1 = komut.ExecuteReader();
            while (dr1.Read())
            {
                lblAdiSoyadi.Text = dr1[0].ToString() + " "
                 + dr1[1].ToString();
            }
            bgl.baglanti().Close();
        }

        private void TabloGuncelle()
        {
            DataTable dt = new DataTable();
            SqlDataAdapter da = new SqlDataAdapter("select
             Randevuid,RandevuTarih,RandevuSaat,Sikayet 
             from Tbl_Randevular where Bransid IN (select 
             Bransid FROM Tbl_Branslar WHERE BransAd LIKE 
             '" + CmbBrans.Text + "') and Doktorid IN 
             (select Doktorid FROM Tbl_Doktorlar WHERE 
             DoktorAd + ' ' + DoktorSoyad LIKE '" + 
             CmbDoktor.Text + "' and 
             RandevuDurum='False') ", bgl.baglanti());
            da.Fill(dt);
            dataGridView2.DataSource = dt;
        }

        private void CmbBrans_SelectedIndexChanged_1(object
         sender, EventArgs e)
        {
            CmbDoktor.Items.Clear();
            SqlCommand komut3 = new SqlCommand("SELECT
             DoktorAd,DoktorSoyad FROM Tbl_Doktorlar WHERE
              Bransid IN (select Bransid from tbl_Branslar 
              where BransAd LIKE @p1)", bgl.baglanti());
            komut3.Parameters.AddWithValue("@p1", 
            CmbBrans.Text);
            SqlDataReader dr3 = komut3.ExecuteReader();
            while (dr3.Read())
            {
                CmbDoktor.Items.Add(dr3[0] + " " + dr3[1]);
            }
            bgl.baglanti().Close();
        }

        private void CmbDoktor_SelectedIndexChanged(object 
        sender, EventArgs e)
        {
            DataTable dt = new DataTable();
            SqlDataAdapter da = new SqlDataAdapter("select
             Randevuid,RandevuTarih,RandevuSaat,Sikayet 
             from Tbl_Randevular where Bransid IN (select 
             Bransid FROM Tbl_Branslar WHERE BransAd LIKE 
             '" + CmbBrans.Text + "') and Doktorid IN 
             (select Doktorid FROM Tbl_Doktorlar WHERE 
             DoktorAd + ' ' + DoktorSoyad LIKE '" + 
             CmbDoktor.Text + "' and 
             RandevuDurum='False') ", bgl.baglanti());
            da.Fill(dt);
            dataGridView2.DataSource = dt;
        }

        private void BtnRandevuAl_Click(object sender, 
        EventArgs e)
        {
            SqlCommand komut = new SqlCommand("UPDATE 
            Tbl_Randevular SET RandevuDurum='True',Hastaid=
            (SELECT Hastaid FROM Tbl_Hastalar WHERE HastaTC
             ='"+ txtHastaTC.Text +"'),Sikayet=@r2 WHERE 
             Randevuid='" + Txtid.Text + "' ", 
             bgl.baglanti());
            komut.Parameters.AddWithValue("@r2", 
            RchSikayet.Text);
            komut.ExecuteNonQuery();
            bgl.baglanti().Close();
            MessageBox.Show("Randevu oluşturuldu.", 
            "Bilgi", MessageBoxButtons.OK, 
            MessageBoxIcon.Information);
            TabloGuncelle();
        }

        private void dataGridView2_CellClick(object sender,
         DataGridViewCellEventArgs e)
        {
            int secilen = dataGridView2.SelectedCells[0].
            RowIndex;
            Txtid.Text = dataGridView2.Rows[secilen].
            Cells[0].Value.ToString(); 
        }

        private void btnRandevularıGetir_Click(object 
        sender, EventArgs e)
        {
            DataTable dt = new DataTable();
            SqlDataAdapter da = new SqlDataAdapter("select
             Randevuid,RandevuTarih,RandevuSaat,Sikayet 
             from Tbl_Randevular where Hastaid=(SELECT 
             Hastaid FROM Tbl_Hastalar WHERE HastaTC='"+ 
             txtHastaTC.Text +"') ", bgl.baglanti());
            da.Fill(dt);
            dataGridView2.DataSource = dt;
        }

        private void btnRandevuSil_Click(object sender, 
        EventArgs e)
        {
            SqlCommand komut = new SqlCommand("UPDATE 
            Tbl_Randevular SET RandevuDurum='False' 
            where Randevuid=@p1", bgl.baglanti());
            komut.Parameters.AddWithValue("@p1", 
            Txtid.Text);
            komut.ExecuteNonQuery();
            bgl.baglanti().Close();
            MessageBox.Show("Randevu Silindi.", "Bilgi",
             MessageBoxButtons.OK, 
             MessageBoxIcon.Information);
            TabloGuncelle();
        }

        private void button1_Click(object sender, EventArgs e)
        {
            Application.Exit();
        }
    }
}
\end{lstlisting}

\textbf{Dosya 13:} Yönetici Detay Formu:

\begin{lstlisting}
using System;
using System.Collections.Generic;
using System.ComponentModel;
using System.Data;
using System.Drawing;
using System.Linq;
using System.Text;
using System.Threading.Tasks;
using System.Windows.Forms;
using System.Data.SqlClient;

namespace Yonetim_Hastane
{
    public partial class FrmYoneticiDetay : Form
    {
        public FrmYoneticiDetay()
        {
            InitializeComponent();
        }

        public String YoneticiTC;
        SqlBaglantisi bgl = new SqlBaglantisi();

        private void FrmYoneticiDetay_Load(object sender, 
        EventArgs e)
        {
            LblTC.Text = YoneticiTC;
            SqlCommand komut = new SqlCommand("select
             YoneticiAd,YoneticiSoyad from Tbl_Yoneticiler 
             where YoneticiTC=@p1", bgl.baglanti());
            komut.Parameters.AddWithValue("@p1", 
            YoneticiTC);
            SqlDataReader dr = komut.ExecuteReader();
            while (dr.Read())
            {
                LblAdSoyad.Text = dr[0] + " " + dr[1];
            }
            bgl.baglanti().Close();
        }

        private void LnkBilgiDuzenle_LinkClicked(object 
        sender, LinkLabelLinkClickedEventArgs e)
        {
            FrmYoneticiBilgiDuzenle fr = new 
            FrmYoneticiBilgiDuzenle();
            fr.TCno = LblTC.Text;
            fr.Show();
        }

        private void label1_Click(object sender, 
        EventArgs e)
        {

        }

        private void BtnDoktorPaneli_Click(object sender, 
        EventArgs e)
        {
            FrmDoktorPaneli drp = new FrmDoktorPaneli();
            drp.Show();
        }

        private void BtnBransPaneli_Click(object sender, 
        EventArgs e)
        {
            FrmBransPaneli frb = new FrmBransPaneli();
            frb.Show();
        }

        private void BtnRandevuListe_Click(object sender, 
        EventArgs e)
        {
            FrmRandevuListesi rl = new FrmRandevuListesi();
            rl.Show();
        }

        private void button1_Click(object sender, 
        EventArgs e)
        {
            FrmDuyurular frd = new FrmDuyurular();
            frd.Show();
        }

        private void linkYoneticiEkle_LinkClicked(object 
        sender, LinkLabelLinkClickedEventArgs e)
        {
            FrmYoneticiEkle fr = new FrmYoneticiEkle();
            fr.Show();
        }

        private void linkSekreterEkle_LinkClicked(object 
        sender, LinkLabelLinkClickedEventArgs e)
        {
            FrmSekreterEkle fr = new FrmSekreterEkle();
            fr.Show();
        }

        private void button2_Click(object sender, 
        EventArgs e)
        {
            Application.Exit();
        }
    }
}
\end{lstlisting}

\textbf{Dosya 14:} Yönetici Bilgi Düzenle Formu:

\begin{lstlisting}
using System;
using System.Collections.Generic;
using System.ComponentModel;
using System.Data;
using System.Drawing;
using System.Linq;
using System.Text;
using System.Threading.Tasks;
using System.Windows.Forms;
using System.Data.SqlClient;

namespace Yonetim_Hastane
{
    public partial class FrmYoneticiBilgiDuzenle : Form
    {
        public FrmYoneticiBilgiDuzenle()
        {
            InitializeComponent();
        }
        public string TCno;

        SqlBaglantisi bgl = new SqlBaglantisi();

        private void BtnBilgiGuncelle_Click(object sender, 
        EventArgs e)
        {
            if (FrmHastaKayit.TCKontrol(MskTC.Text)) {
                SqlCommand komut2 = new SqlCommand("update
                 Tbl_Yoneticiler set YoneticiAd=@p1, 
                 YoneticiSoyad=@p2, YoneticiSifre=@p3 where
                  YoneticiTC=@p4", bgl.baglanti());
                komut2.Parameters.AddWithValue("@p1", 
                TxtAd.Text);
                komut2.Parameters.AddWithValue("@p2", 
                TxtSoyad.Text);
                komut2.Parameters.AddWithValue("@p3", 
                TxtSifre.Text);
                komut2.Parameters.AddWithValue("@p4", 
                MskTC.Text);
                komut2.ExecuteNonQuery();
                bgl.baglanti().Close();
                MessageBox.Show("Bilgileriniz güncellendi."
                , "Bilgi", MessageBoxButtons.OK,
                 MessageBoxIcon.Information);
            }
        }

        private void FrmYoneticiBilgiDuzenle_Load(object 
        sender, EventArgs e)
        {
            MskTC.Text = TCno;
            SqlCommand komut = new SqlCommand("select * 
            from Tbl_Yoneticiler where YoneticiTC=@p1", 
            bgl.baglanti());
            komut.Parameters.AddWithValue("@p1", TCno);
            SqlDataReader dr = komut.ExecuteReader();
            while (dr.Read())
            {
                TxtAd.Text = dr[2].ToString();
                TxtSoyad.Text = dr[3].ToString();
                TxtSifre.Text = dr[4].ToString();
            }
            bgl.baglanti().Close();
        }
    }
}
\end{lstlisting}

\textbf{Dosya 15:} Yönetici Ekle Formu:

\begin{lstlisting}
using System;
using System.Collections.Generic;
using System.ComponentModel;
using System.Data;
using System.Drawing;
using System.Linq;
using System.Text;
using System.Threading.Tasks;
using System.Windows.Forms;
using System.Data.SqlClient;

namespace Yonetim_Hastane
{
    public partial class FrmYoneticiEkle : Form
    {
        public FrmYoneticiEkle()
        {
            InitializeComponent();
        }

        SqlBaglantisi bgl = new SqlBaglantisi();
        private void BtnYoneticiEkle_Click(object sender, 
        EventArgs e)
        {
            if (FrmHastaKayit.TCKontrol(MskTC.Text)) {
                SqlCommand komut2 = new SqlCommand("insert 
                into
                 Tbl_Yoneticiler(YoneticiTC,YoneticiAd,
                 YoneticiSoyad,YoneticiSifre) values 
                 (@p1, @p2, @p3, @p4)", bgl.baglanti());
                komut2.Parameters.AddWithValue("@p2", 
                TxtAd.Text);
                komut2.Parameters.AddWithValue("@p3", 
                TxtSoyad.Text);
                komut2.Parameters.AddWithValue("@p4", 
                TxtSifre.Text);
                komut2.Parameters.AddWithValue("@p1", 
                MskTC.Text);
                komut2.ExecuteNonQuery();
                bgl.baglanti().Close();
                MessageBox.Show("Yönetici Eklendi", 
                "Bilgi", MessageBoxButtons.OK,
                 MessageBoxIcon.Information);
                TabloGuncelle();
            }
        }

        private void TabloGuncelle()
        {
            DataTable dt2 = new DataTable();
            SqlDataAdapter da2 = new SqlDataAdapter("select
             Yoneticiid,YoneticiAd,YoneticiSoyad,YoneticiTC
             ,YoneticiSifre from Tbl_Yoneticiler", 
             bgl.baglanti());
            da2.Fill(dt2);
            dataGridView1.DataSource = dt2;
        }

        private void dataGridView1_CellClick(object sender,
         DataGridViewCellEventArgs e)
        {
            int secilen = dataGridView1.SelectedCells[0].
            RowIndex;
            TxtYoneticiid.Text = dataGridView1.
            Rows[secilen].Cells[0].Value.ToString();
            TxtAd.Text = dataGridView1.Rows[secilen].
            Cells[1].Value.ToString();
            TxtSoyad.Text = dataGridView1.Rows[secilen].
            Cells[2].Value.ToString();
            MskTC.Text = dataGridView1.Rows[secilen].
            Cells[3].Value.ToString();
            TxtSifre.Text = dataGridView1.Rows[secilen].
            Cells[4].Value.ToString();
        }

        private void btnYoneticiSil_Click(object sender, 
        EventArgs e)
        {
            SqlCommand komut2 = new SqlCommand("delete from
             Tbl_Yoneticiler where Yoneticiid=@p1", 
             bgl.baglanti());
            komut2.Parameters.AddWithValue("@p1", 
            TxtYoneticiid.Text);
            komut2.ExecuteNonQuery();
            bgl.baglanti().Close();
            MessageBox.Show("Yönetici Silindi.", "Bilgi",
             MessageBoxButtons.OK, 
             MessageBoxIcon.Information);
            TabloGuncelle();
        }

        private void FrmYoneticiEkle_Load(object sender, 
        EventArgs e)
        {
            DataTable dt2 = new DataTable();
            SqlDataAdapter da2 = new SqlDataAdapter("select
             Yoneticiid,YoneticiAd,YoneticiSoyad,YoneticiTC
             ,YoneticiSifre from Tbl_Yoneticiler", 
             bgl.baglanti());
            da2.Fill(dt2);
            dataGridView1.DataSource = dt2;
        }

        private void button1_Click(object sender, 
        EventArgs e)
        {
            SqlCommand komut2 = new SqlCommand("update 
            Tbl_Yoneticiler set YoneticiAd=@p1, 
            YoneticiSoyad=@p2, YoneticiTC=@p3, 
            YoneticiSifre=@p4 where Yoneticiid=@p5", 
            bgl.baglanti());
            komut2.Parameters.AddWithValue("@p1", 
            TxtAd.Text);
            komut2.Parameters.AddWithValue("@p2", 
            TxtSoyad.Text);
            komut2.Parameters.AddWithValue("@p3", 
            MskTC.Text);
            komut2.Parameters.AddWithValue("@p4", 
            TxtSifre.Text);
            komut2.Parameters.AddWithValue("@p5", 
            TxtYoneticiid.Text);

            komut2.ExecuteNonQuery();
            bgl.baglanti().Close();
            MessageBox.Show("Yönetici Güncellendi.", "Bilgi",
             MessageBoxButtons.OK, 
             MessageBoxIcon.Information);
            TabloGuncelle();
           
        }
    }
}
\end{lstlisting}

\textbf{Dosya 16:} Doktor Paneli Formu:

\begin{lstlisting}
using System;
using System.Collections.Generic;
using System.ComponentModel;
using System.Data;
using System.Drawing;
using System.Linq;
using System.Text;
using System.Threading.Tasks;
using System.Windows.Forms;
using System.Data.SqlClient;

namespace Yonetim_Hastane
{
    public partial class FrmDoktorPaneli : Form
    {
        public FrmDoktorPaneli()
        {
            InitializeComponent();
        }

        SqlBaglantisi bgl = new SqlBaglantisi();

        private void FrmDoktorPaneli_Load(object sender, 
        EventArgs e)
        {
            //Doktorları Datagridview2'e aktarma
            DataTable dt1 = new DataTable();
            SqlDataAdapter da1 = new SqlDataAdapter("select
             Tbl_Doktorlar.Doktorid, Tbl_Doktorlar.DoktorAd
             , Tbl_Doktorlar.DoktorSoyad, Tbl_Branslar.
             BransAd, Tbl_Doktorlar.DoktorTC, Tbl_Doktorlar
             .DoktorSifre from Tbl_Doktorlar, Tbl_Branslar 
             WHERE Tbl_Doktorlar.Bransid = Tbl_Branslar.
             Bransid", bgl.baglanti());
            da1.Fill(dt1);
            dataGridView1.DataSource = dt1;

            //Branşları Combobox'a aktarma
            SqlCommand komut1 = new SqlCommand("select 
            BransAd from Tbl_Branslar", bgl.baglanti());
            SqlDataReader dr1 = komut1.ExecuteReader();
            while (dr1.Read())
            {
                CmbBrans.Items.Add(dr1[0]);
            }
            bgl.baglanti().Close();
        }

        private void TabloGuncelle()
        {
            //Tablo güncelleme
            DataTable dt1 = new DataTable();
            SqlDataAdapter da1 = new SqlDataAdapter("select
             Tbl_Doktorlar.Doktorid, Tbl_Doktorlar.DoktorAd
             , Tbl_Doktorlar.DoktorSoyad, Tbl_Branslar.
             BransAd, Tbl_Doktorlar.DoktorTC, Tbl_Doktorlar
             .DoktorSifre from Tbl_Doktorlar, Tbl_Branslar 
             WHERE Tbl_Doktorlar.Bransid = Tbl_Branslar.
             Bransid", bgl.baglanti());
            da1.Fill(dt1);
            dataGridView1.DataSource = dt1;
        }

        private void BtnEkle_Click(object sender, 
        EventArgs e)
        {
            if (FrmHastaKayit.TCKontrol(MskTC.Text)) {
                SqlCommand komut = new SqlCommand("insert 
                into
                 Tbl_Doktorlar(DoktorAd,DoktorSoyad,Bransid
                 ,DoktorTC,DoktorSifre) values(@d1,@d2,
                 (SELECT Bransid FROM Tbl_Branslar WHERE 
                 BransAd='" + CmbBrans.Text + "'),
                 @d4,@d5)", bgl.baglanti());
                komut.Parameters.AddWithValue("@d1", 
                TxtAd.Text);
                komut.Parameters.AddWithValue("@d2", 
                TxtSoyad.Text);
                komut.Parameters.AddWithValue("@d4", 
                MskTC.Text);
                komut.Parameters.AddWithValue("@d5", 
                TxtSifre.Text);
                komut.ExecuteNonQuery();
                bgl.baglanti().Close();
                MessageBox.Show("Doktor eklendi.", "Bilgi",
                 MessageBoxButtons.OK, 
                 MessageBoxIcon.Information);
                TabloGuncelle();
            }
        }

        private void dataGridView1_CellClick(object sender,
         DataGridViewCellEventArgs e)
        {
            int secilen = dataGridView1.SelectedCells[0].
            RowIndex;
            TxtDoktorid.Text = dataGridView1.Rows[secilen].
            Cells[0].Value.ToString();
            TxtAd.Text = dataGridView1.Rows[secilen].
            Cells[1].Value.ToString();
            TxtSoyad.Text = dataGridView1.Rows[secilen].
            Cells[2].Value.ToString();
            CmbBrans.Text = dataGridView1.Rows[secilen].
            Cells[3].Value.ToString();
            MskTC.Text = dataGridView1.Rows[secilen].
            Cells[4].Value.ToString();
            TxtSifre.Text = dataGridView1.Rows[secilen].
            Cells[5].Value.ToString();
        }

        private void BtnSil_Click(object sender, 
        EventArgs e)
        {
            SqlCommand komut2 = new SqlCommand("DELETE FROM
             Tbl_Randevular WHERE Doktorid='" + 
             TxtDoktorid.Text + "'", bgl.baglanti());
            komut2.ExecuteNonQuery();
            bgl.baglanti().Close();

            SqlCommand komut = new SqlCommand("DELETE FROM 
            Tbl_Doktorlar WHERE Doktorid='"+ 
            TxtDoktorid.Text + "'", bgl.baglanti());
            komut.ExecuteNonQuery();
            bgl.baglanti().Close();
            MessageBox.Show("Doktor kaydı silindi.",
            "Bilgi",MessageBoxButtons.OK,
            MessageBoxIcon.Asterisk);
            TabloGuncelle();
        }

        private void BtnGuncelle_Click(object sender, 
        EventArgs e)
        {
            if (FrmHastaKayit.TCKontrol(MskTC.Text))
            {
                SqlCommand komut = new SqlCommand("update 
                Tbl_Doktorlar set DoktorAd=@d2, 
                DoktorSoyad=@d3, Bransid=(SELECT Bransid 
                FROM Tbl_Branslar WHERE BransAd=@d4), 
                DoktorTC=@d5, DoktorSifre=@d6 where 
                Doktorid=@d1", bgl.baglanti());
                komut.Parameters.AddWithValue("@d1", 
                TxtDoktorid.Text);
                komut.Parameters.AddWithValue("@d2", 
                TxtAd.Text);
                komut.Parameters.AddWithValue("@d3", 
                TxtSoyad.Text);
                komut.Parameters.AddWithValue("@d4", 
                CmbBrans.Text);
                komut.Parameters.AddWithValue("@d5", 
                MskTC.Text);
                komut.Parameters.AddWithValue("@d6", 
                TxtSifre.Text);

                komut.ExecuteNonQuery();
                bgl.baglanti().Close();
                MessageBox.Show("Doktor bilgisi 
                güncellendi.", "Bilgi", 
                MessageBoxButtons.OK, 
                MessageBoxIcon.Information);

                TabloGuncelle();
            }
        }
    }
}
\end{lstlisting}

\textbf{Dosya 17:} Branş Paneli Formu:

\begin{lstlisting}
using System;
using System.Collections.Generic;
using System.ComponentModel;
using System.Data;
using System.Drawing;
using System.Linq;
using System.Text;
using System.Threading.Tasks;
using System.Windows.Forms;
using System.Data.SqlClient;

namespace Yonetim_Hastane
{
    public partial class FrmBransPaneli : Form
    {
        public FrmBransPaneli()
        {
            InitializeComponent();
        }

        SqlBaglantisi bgl = new SqlBaglantisi();

        private void FrmBransPaneli_Load(object sender, 
        EventArgs e)
        {
            DataTable dt = new DataTable();
            SqlDataAdapter da = new SqlDataAdapter("select 
            Bransid, BransAd from Tbl_Branslar", 
            bgl.baglanti());
            da.Fill(dt);
            dataGridView1.DataSource = dt;
        }

        private void TabloGuncelle()
        {
            DataTable dt = new DataTable();
            SqlDataAdapter da = new SqlDataAdapter("select 
            Bransid, BransAd from Tbl_Branslar", 
            bgl.baglanti());
            da.Fill(dt);
            dataGridView1.DataSource = dt;
        }

        private void BtnEkle_Click(object sender, 
        EventArgs e)
        {
            SqlCommand komut = new SqlCommand("insert into
             Tbl_Branslar(BransAd) values(@b1)", 
             bgl.baglanti());
            komut.Parameters.AddWithValue("@b1", 
            TxtBrans.Text);
            komut.ExecuteNonQuery();
            bgl.baglanti().Close();
            TabloGuncelle();
            MessageBox.Show("Branş eklendi.", "Bilgi",
             MessageBoxButtons.OK, 
             MessageBoxIcon.Information); 
        }

        private void dataGridView1_CellClick(object sender,
         DataGridViewCellEventArgs e)
        {
            int secilen = dataGridView1.SelectedCells[0].
            RowIndex;
            Txtid.Text = dataGridView1.Rows[secilen].
            Cells[0].Value.ToString();
            TxtBrans.Text = dataGridView1.Rows[secilen].
            Cells[1].Value.ToString();
        }

        private void BtnSil_Click(object sender, 
        EventArgs e)
        {
            SqlCommand komut = new SqlCommand("delete from 
            Tbl_Branslar where Bransid=@b1", 
            bgl.baglanti());
            komut.Parameters.AddWithValue("@b1", 
            Txtid.Text);
            komut.ExecuteNonQuery();
            bgl.baglanti().Close();
            TabloGuncelle();
            MessageBox.Show("Branş kaydı silindi.", 
            "Bilgi", MessageBoxButtons.OK, 
            MessageBoxIcon.Information);   
        }

        private void BtnGuncelle_Click(object sender, 
        EventArgs e)
        {
            SqlCommand komut = new SqlCommand("update 
            Tbl_Branslar set BransAd=@b1 where 
            Bransid=@b2", bgl.baglanti());
            komut.Parameters.AddWithValue("@b1", 
            TxtBrans.Text);
            komut.Parameters.AddWithValue("@b2", 
            Txtid.Text);
            komut.ExecuteNonQuery();
            bgl.baglanti().Close();
            TabloGuncelle();
            MessageBox.Show("Branş kaydı güncellendi.", 
            "Bilgi", MessageBoxButtons.OK, 
            MessageBoxIcon.Information);
        }

        private void button1_Click(object sender, 
        EventArgs e)
        {
            Application.Exit();
        }
    }
}
\end{lstlisting}

\textbf{Dosya 18:} Program.cs dosyası:

\begin{lstlisting}
using System;
using System.Collections.Generic;
using System.Linq;
using System.Threading.Tasks;
using System.Windows.Forms;

namespace Yonetim_Hastane
{
    static class Program
    {
        /// <summary>
        /// The main entry point for the application.
        /// </summary>
        [STAThread]
        static void Main()
        {
            Application.EnableVisualStyles();
            Application.
            SetCompatibleTextRenderingDefault(false);
            Application.Run(new FrmGirisler());
        }
    }
}
\end{lstlisting}

Programda toplam 2154 satır kod vardır.